\documentclass[12pt]{ctexart}
\usepackage[utf8]{inputenc}
\usepackage{graphicx}
\usepackage{float}
\usepackage{booktabs}
\usepackage{amsmath}
\usepackage{hyperref}
\usepackage{listings}
\usepackage{xcolor}
\usepackage{caption}
\usepackage{longtable}

% 代码样式设置
\definecolor{codegreen}{rgb}{0,0.6,0}
\definecolor{codegray}{rgb}{0.5,0.5,0.5}
\definecolor{codepurple}{rgb}{0.58,0,0.82}
\definecolor{backcolour}{rgb}{0.95,0.95,0.92}

\lstdefinestyle{python}{
    backgroundcolor=\color{backcolour},   
    commentstyle=\color{codegreen},
    keywordstyle=\color{magenta},
    numberstyle=\tiny\color{codegray},
    stringstyle=\color{codepurple},
    basicstyle=\ttfamily\footnotesize,
    breakatwhitespace=false,         
    breaklines=true,                 
    captionpos=b,                    
    keepspaces=true,                 
    numbers=left,                    
    numbersep=5pt,                  
    showspaces=false,                
    showstringspaces=false,
    showtabs=false,                  
    tabsize=2,
    language=Python
}

\lstdefinestyle{R}{
    backgroundcolor=\color{backcolour},   
    commentstyle=\color{codegreen},
    keywordstyle=\color{magenta},
    numberstyle=\tiny\color{codegray},
    stringstyle=\color{codepurple},
    basicstyle=\ttfamily\footnotesize,
    breakatwhitespace=false,         
    breaklines=true,                 
    captionpos=b,                    
    keepspaces=true,                 
    numbers=left,                    
    numbersep=5pt,                  
    showspaces=false,                
    showstringspaces=false,
    showtabs=false,                  
    tabsize=2,
    language=R
}

\title{数据分析报告}
\author{董泓麟\quad 李嘉俊}
\date{\today}

\begin{document}

\maketitle

\begin{abstract}
本报告对数据集进行了全面的分析,包括描述性统计、可视化分析、假设检验和数据建模。通过Python和R语言相结合的方式,深入挖掘数据特征和规律。
\end{abstract}

\tableofcontents
\newpage

\section{引言}

\subsection{研究背景}
糖尿病是一种常见的慢性疾病,对全球公共卫生造成重大负担。根据世界卫生组织数据,糖尿病患病率逐年上升,导致心血管疾病、肾脏损伤等并发症。在美国,糖尿病影响约 10\% 的人口,早期预测和干预至关重要。本研究旨在通过分析行为风险因素监测系统(BRFSS)数据,探索关键变量对糖尿病发生的影响,为预防策略提供数据支持。重点变量包括 BMI、心理健康天数、身体健康天数等,结合分类变量如高血压、吸烟等,进行描述性统计、可视化和假设检验。

\subsection{数据来源}
数据来源于美国疾病控制与预防中心(CDC)的行为风险因素监测系统(BRFSS),这是一个年度电话调查,收集美国各州居民的健康行为和疾病信息。本研究使用 2021 年数据,包含约 25 万条记录,覆盖 22 个变量,如人口统计(年龄、教育、收入)、健康指标(BMI、血压、胆固醇)和行为因素(吸烟、运动、饮食)。数据经过预处理,去除缺失值和异常值,确保分析可靠性。BRFSS 数据公开可用,代表美国成人人口分布,为流行病学研究提供有力依据。

\section{数据预处理}

\subsection{数据加载与清洗}
本节描述数据预处理的过程和方法,包括数据加载、初步探索、缺失值处理以及异常值检测。通过数据分析,发现原始数据集无缺失值,确保分析的完整性和可靠性。

\textbf{关键步骤:}
\begin{itemize}
    \item \textbf{数据加载和初步探索}:使用 Python 的 pandas 库加载 CSV 文件,进行形状检查(253680 行,22 列)和基本统计描述,确认变量类型和分布。
    \item \textbf{缺失值处理}:检查各变量缺失率,发现无缺失值,无需填充或删除操作。
    \item \textbf{异常值检测和处理}:通过箱线图和统计方法检测离群值,对连续变量(如 BMI)进行 IQR 方法处理,剔除极端值以提升模型稳定性。
\end{itemize}

\subsection{数据探索}
数据探索旨在了解数据集的基本结构、变量分布和潜在模式。通过初步分析,发现数据质量高,无缺失值,适合后续建模。

\textbf{数据集基本信息:}
\begin{itemize}
    \item 数据形状:253680 行,22 列。
    \item 变量类型:包含 19 个分类变量(二元/序数,如 HighBP、Education)和 3 个连续变量(BMI、MentHlth、PhysHlth)。
    \item 目标变量:Diabetes\_binary(糖尿病发生,0=无,1=有)。
\end{itemize}

\textbf{目标变量分布:}
\begin{itemize}
    \item 0: 218334 (86.07\%)
    \item 1: 35346 (13.93\%)
\end{itemize}
数据略不平衡,但仍可用于分析。

\textbf{分类变量概述:} 大多数分类变量(如 HighBP、Smoker)为二元,分布相对均匀(见表 \ref{tab:class_stats})。可视化分析(见第 4 节)进一步展示分布特征。

\section{描述性统计}
\subsection{数值型变量统计}
展示数值型变量的集中趋势、离散程度和分布形态。

\textbf{主要统计量:}
\begin{itemize}
    \item \textbf{均值 (Mean)}:数据集的平均值,计算公式为 \(\bar{x} = \frac{1}{n} \sum_{i=1}^{n} x_i\),其中 \(n\) 为样本大小,\(x_i\) 为第 \(i\) 个观测值。
    \item \textbf{中位数 (Median)}:将数据集排序后,位于中间位置的值。对于奇数 \(n\),为第 \(\frac{n+1}{2}\) 个值;对于偶数 \(n\),为中间两个值的平均。
    \item \textbf{众数 (Mode)}:数据集中出现频率最高的值。如果有多个相同频率的值,则有多个众数;如果所有值频率相等,则无众数。
    \item \textbf{标准差 (Standard Deviation)}:衡量数据离散程度的指标,计算公式为 \(\sigma = \sqrt{\frac{1}{n} \sum_{i=1}^{n} (x_i - \bar{x})^2}\),其中 \(\bar{x}\) 为均值。
    \item \textbf{方差 (Variance)}:标准差的平方,计算公式为 \(\sigma^2 = \frac{1}{n} \sum_{i=1}^{n} (x_i - \bar{x})^2\),反映数据的波动幅度。
    \item \textbf{极差 (Range)}:数据集的最大值与最小值之差,计算公式为 \(R = x_{\max} - x_{\min}\),简单衡量数据跨度。
    \item \textbf{偏度 (Skewness)}:衡量数据分布对称性的指标,正偏表示右尾长,负偏表示左尾长,计算公式为 \(\gamma_1 = \frac{\sum_{i=1}^{n} (x_i - \bar{x})^3 / n}{\left( \sum_{i=1}^{n} (x_i - \bar{x})^2 / n \right)^{3/2}}\)。
    \item \textbf{峰度 (Kurtosis)}:衡量数据分布尾部厚度的指标,相对于正态分布的峰度为 0,计算公式为 \(\gamma_2 = \frac{\sum_{i=1}^{n} (x_i - \bar{x})^4 / n}{\left( \sum_{i=1}^{n} (x_i - \bar{x})^2 / n \right)^2} - 3\)。
\end{itemize}

\begin{table}[H]
\centering
\small
\caption{数值型变量主要统计量}
\label{tab:numeric_stats}
\begin{tabular}{p{3.2cm}ccccccc}
\toprule
Variable & Median & Mode & Std & Var & Range & Skew & Kurt \\
\midrule
BMI & 27.0 & 27.0 & 6.61 & 43.67 & 86.0 & 2.12 & 11.0 \\
MentHlth & 0.0 & 0.0 & 7.41 & 54.95 & 30.0 & 2.72 & 6.44 \\
PhysHlth & 0.0 & 0.0 & 8.72 & 76.0 & 30.0 & 2.21 & 3.5 \\
Age & 8.0 & 9.0 & 3.05 & 9.33 & 12.0 & -0.36 & -0.58 \\
\bottomrule
\end{tabular}
\end{table}

\subsection{分类变量统计}
展示分类变量的频数分布和比例。

\begin{longtable}{lccc}
\caption{分类变量频数分布汇总}\label{tab:class_stats} \\
\toprule
变量 & 类别 & 频数 & 比例 (\%) \\
\midrule
\endfirsthead

\multicolumn{4}{c}%
{{\bfseries 续表 \thetable\ 分类变量频数分布汇总}} \\
\toprule
变量 & 类别 & 频数 & 比例 (\%) \\
\midrule
\endhead

\midrule
\multicolumn{4}{r}{{续下页}} \\
\endfoot

\midrule
\endlastfoot

HighBP & 0 & 144851 & 57.10 \\
 & 1 & 108829 & 42.90 \\
\midrule
HighChol & 0 & 146089 & 57.59 \\
 & 1 & 107591 & 42.41 \\
\midrule
CholCheck & 1 & 244210 & 96.27 \\
 & 0 & 9470 & 3.73 \\
\midrule
Smoker & 0 & 141257 & 55.68 \\
 & 1 & 112423 & 44.32 \\
\midrule
Stroke & 0 & 243388 & 95.94 \\
 & 1 & 10292 & 4.06 \\
\midrule
HeartDiseaseorAttack & 0 & 229787 & 90.58 \\
 & 1 & 23893 & 9.42 \\
\midrule
PhysActivity & 1 & 191920 & 75.65 \\
 & 0 & 61760 & 24.35 \\
\midrule
Fruits & 1 & 160898 & 63.43 \\
 & 0 & 92782 & 36.57 \\
\midrule
Veggies & 1 & 205841 & 81.14 \\
 & 0 & 47839 & 18.86 \\
\midrule
HvyAlcoholConsump & 0 & 239424 & 94.38 \\
 & 1 & 14256 & 5.62 \\
\midrule
AnyHealthcare & 1 & 241263 & 95.11 \\
 & 0 & 12417 & 4.89 \\
\midrule
NoDocbcCost & 0 & 232326 & 91.58 \\
 & 1 & 21354 & 8.42 \\
\midrule
GenHlth & 2 & 89084 & 35.12 \\
 & 3 & 75646 & 29.82 \\
 & 1 & 45299 & 17.86 \\
 & 4 & 31570 & 12.44 \\
 & 5 & 12081 & 4.76 \\
\midrule
DiffWalk & 0 & 211005 & 83.18 \\
 & 1 & 42675 & 16.82 \\
\midrule
Sex & 0 & 141974 & 55.97 \\
 & 1 & 111706 & 44.03 \\
\midrule
Education & 6 & 107325 & 42.31 \\
 & 5 & 69910 & 27.56 \\
 & 4 & 62750 & 24.74 \\
 & 3 & 9478 & 3.74 \\
 & 2 & 4043 & 1.59 \\
 & 1 & 174 & 0.07 \\
\midrule
Income & 8 & 90385 & 35.63 \\
 & 7 & 43219 & 17.04 \\
 & 6 & 36470 & 14.38 \\
 & 5 & 25883 & 10.20 \\
 & 4 & 20135 & 7.94 \\
 & 3 & 15994 & 6.30 \\
 & 2 & 11783 & 4.64 \\
 & 1 & 9811 & 3.87 \\

\end{longtable}

\clearpage
\section{可视化分析}

\subsection{单变量分析}
通过直方图、箱线图等展示单个变量的分布特征。单变量分析有助于理解数据的集中趋势、离散程度和分布形态。例如,对于连续变量,可以观察是否接近正态分布(公式:\(f(x) = \frac{1}{\sigma \sqrt{2\pi}} e^{-\frac{(x-\mu)^2}{2\sigma^2}}\),其中 \(\mu\) 为均值,\(\sigma\) 为标准差)。对于分类变量,可以检查类别平衡性。

\begin{figure}[H]
    \centering
    \includegraphics[width=0.9\textwidth]{../figs/univariate/BMI\_hist.png}
    \caption{BMI 分布直方图与核密度估计。观察到分布略右偏(正偏度),多数样本 BMI 在 20-35 范围内,符合健康人群特征。}
    \label{fig:bmi_hist}
\end{figure}

\begin{figure}[H]
    \centering
    \includegraphics[width=0.7\textwidth]{../figs/univariate/MentHlth\_box.png}
    \caption{MentHlth 箱线图。显示中位数为 0,存在较多异常值(上四分位数外),表明心理健康问题在少数样本中突出。}
    \label{fig:menthlth_box}
\end{figure}

\begin{figure}[H]
    \centering
    \includegraphics[width=0.7\textwidth]{../figs/univariate/HighBP\_bar.png}
    \caption{HighBP 频数条形图。类别 0(无高血压)占比约 60\%,类别 1(有高血压)占比 40\%,数据相对平衡。}
    \label{fig:highbp_bar}
\end{figure}

\subsection{多变量分析}
通过散点图矩阵、相关性热力图等展示变量间的关系。多变量分析揭示变量相关性,例如 Pearson 相关系数(公式:\(r = \frac{\sum (x_i - \bar{x})(y_i - \bar{y})}{\sqrt{\sum (x_i - \bar{x})^2 \sum (y_i - \bar{y})^2}}\),范围 [-1, 1],绝对值越大相关性越强。

\begin{figure}[H]
    \centering
    \includegraphics[width=0.9\textwidth]{../figs/multivariate/BMI\_MentHlth\_PhysHlth\_heatmap.png}
    \caption{数值变量相关性热力图。MentHlth 与 PhysHlth 相关系数约为 0.35(弱相关),其他变量相关性都较小,数据独立性较好。}
    \label{fig:heatmap}
\end{figure}

\begin{figure}[H]
    \centering
    \includegraphics[width=0.7\textwidth]{../figs/multivariate/HighBP\_Diabetes\_binary\_stacked\_bar.png}
    \caption{HighBP 与 Diabetes\_binary 堆叠条形图。高血压人群中糖尿病风险(类别 1)占比约 25\%,高于无高血压人群的 10\%,显示显著关联。}
    \label{fig:stacked_bar}
\end{figure}

\begin{figure}[H]
    \centering
    \includegraphics[width=0.7\textwidth]{../figs/multivariate/BMI\_Diabetes\_binary\_box\_hue.png}
    \caption{BMI vs Diabetes\_binary 箱线图。糖尿病风险组(类别 1)BMI 中位数高于非风险组,证实 BMI 为关键风险因素。}
    \label{fig:box_hue}
\end{figure}


\section{假设检验}

\subsection{非参数检验:Mann-Whitney U 检验}

用于比较两独立组的差异。例如,比较糖尿病组(Diabetes\_binary=1)和非糖尿病组(Diabetes\_binary=0)的 BMI、MentHlth 和 PhysHlth 的中位数差异。

检验假设:

$H_0$ :两组分布相同;

$H_1$ :两组分布不同。

检验统计量:$U = n_1 n_2 + \frac{n_1(n_1+1)}{2} - R_1$,其中 $R_1$ 为第一组秩和,p-value < 0.05 表示显著差异。

\paragraph{检验变量选择说明}
选取 BMI、MentHlth 和 PhysHlth 三个变量进行 Mann-Whitney U 检验,原因是这三个变量的取值范围较大,适合比较其在糖尿病组与非糖尿病组之间的位置参数差异。其余变量多为分类变量,更适合进行比例比较或逻辑回归建模,而不适用于比较位置参数。

\begin{table}[H]
\centering
\small
\caption{Mann-Whitney U 检验结果}
\label{tab:mann_whitney}
\begin{tabular}{lccc}
\toprule
Variable & U Statistic & p-value & Significant \\
\midrule
BMI & 2405335216.50 & 0.000000 & Yes \\
MentHlth & 3648123651.50 & 0.000000 & Yes \\
PhysHlth & 2986457157.00 & 0.000000 & Yes \\
\bottomrule
\end{tabular}
\end{table}

\paragraph{结果解释}
BMI 在糖尿病组中显著高于非糖尿病组(p-value < 0.05),证实 BMI 为关键风险因素。MentHlth 和 PhysHlth 在两组间也存在显著差异。值得注意的是,p-value 极小(趋近 0)是由于样本量巨大(超过 25 万),Mann-Whitney U 检验对大样本差异高度敏感,这并不影响结论的有效性。

\section{数据建模}
\subsection{逻辑回归}
由于目标变量 Diabetes\_binary 为二元分类变量(0=无糖尿病,1=有糖尿病),不适合使用普通线性回归,因此采用逻辑回归进行建模分析。

\paragraph{模型构建与评估}
使用 sklearn 中的 LogisticRegression 模型进行训练,采用 75\% 的数据作为训练集,25\% 作为测试集,并保持类别分布平衡(stratify=y)。模型评估指标包括准确率、AUC 值、混淆矩阵和分类报告。

\paragraph{模型结果}
\begin{itemize}
    \item \textbf{准确率}:0.817
    \item \textbf{AUC}:0.821
    \item \textbf{混淆矩阵}:\\
    真实为“无糖尿病”预测正确 53328 例,误判为“糖尿病” 7444 例;\\
    真实为“糖尿病”预测正确 1393 例,误判为“无糖尿病” 5445 例。
    \item \textbf{关键影响因素(按 OR 值排序)}:\\
    通过计算系数与 OR(Odds Ratio)值,提取前 15 个最有影响力的因素,结果如下(详见附录代码)。
\end{itemize}

\subsection*{ROC 曲线与混淆矩阵可视化}
\begin{figure}[H]
    \centering
    \includegraphics[width=0.7\textwidth]{../figs/model/roc_curve.png}
    \caption{ROC 曲线(AUC = 0.821)}
    \label{fig:roc_curve}
\end{figure}

\begin{figure}[H]
    \centering
    \includegraphics[width=0.7\textwidth]{../figs/model/confusion_matrix.png}
    \caption{混淆矩阵}
    \label{fig:confusion_matrix}
\end{figure}

\subsection{模型解释与临床意义}
逻辑回归模型显示,BMI、高血压(HighBP)、胆固醇(HighChol)等为糖尿病的重要预测因子。OR 值大于 1 的变量表示增加糖尿病风险,小于 1 的变量表示降低风险。模型结果可为临床干预和公共卫生策略提供参考依据。

\subsection{模型局限性}
\begin{itemize}
    \item 数据为横断面调查,无法推断因果关系。
    \item 类别不平衡可能影响少数类别的预测性能。
    \item 模型未考虑变量间的交互作用。
\end{itemize}

\subsection{建议建模方向}
未来可考虑集成学习(如随机森林、XGBoost)或深度学习模型(如LSTM,Unet)进一步提升预测性能,并结合时间序列数据开展纵向研究。

\appendix
\section{代码附录}

\subsection{数据预处理代码}

\subsubsection{Python数据加载与清洗}
\begin{lstlisting}[style=python, caption=Python数据加载与清洗]
import pandas as pd
import numpy as np
import matplotlib.pyplot as plt
import seaborn as sns

# 加载数据
data = pd.read_csv('combined.csv') # 这是数据文件名

# 数据清洗
# 处理缺失值
data = data.dropna()
# 处理异常值
# ... 其他预处理步骤

# 查看数据基本信息
print(data.info())
print(data.describe())
print(data.head())
\end{lstlisting}

\subsubsection{R数据加载与清洗}
\begin{lstlisting}[style=R, caption=R数据加载与清洗]
# 加载数据
data <- read.csv('combined.csv')

# 数据清洗
# 处理缺失值
data <- na.omit(data)
# 处理异常值
# ... 其他预处理步骤

# 查看数据基本信息
summary(data)
head(data)
str(data)
\end{lstlisting}

\subsection{描述性统计代码}

\subsubsection{Python描述性统计}
\begin{lstlisting}[style=python, caption=Python描述性统计]
# 数值型变量的描述性统计
numeric_stats = data.describe()
print(numeric_stats)

# 计算偏度和峰度
from scipy.stats import skew, kurtosis
for column in data.select_dtypes(include=[np.number]).columns:
    print(f"{column}: 偏度={skew(data[column])}, 峰度={kurtosis(data[column])}")

# 分类变量的频数统计
categorical_stats = data.describe(include=['object'])
print(categorical_stats)

# 各分类变量的频数分布
for column in data.select_dtypes(include=['object']).columns:
    print(f"\n{column}的频数分布:")
    print(data[column].value_counts())
\end{lstlisting}

\subsubsection{R描述性统计}
\begin{lstlisting}[style=R, caption=R描述性统计]
# 数值型变量的描述性统计
summary(data)

# 计算偏度和峰度
library(moments)
for(col in names(data)[sapply(data, is.numeric)]){
    cat(col, ": 偏度=", skewness(data[[col]]), 
        ", 峰度=", kurtosis(data[[col]]), "\n")
}

# 分类变量统计
table(data$categorical_variable)
prop.table(table(data$categorical_variable))
\end{lstlisting}

\subsection{可视化分析代码}

\subsubsection{Python单变量可视化}
\begin{lstlisting}[style=python, caption=Python单变量可视化]
# 数值型变量的直方图
plt.figure(figsize=(15, 10))
for i, column in enumerate(data.select_dtypes(include=[np.number]).columns):
    plt.subplot(3, 3, i+1)
    data[column].hist(bins=30)
    plt.title(f'{column}分布')
plt.tight_layout()
plt.show()

# 箱线图
plt.figure(figsize=(15, 10))
data.select_dtypes(include=[np.number]).boxplot()
plt.title('数值型变量箱线图')
plt.xticks(rotation=45)
plt.show()
\end{lstlisting}

\subsubsection{Python多变量可视化}
\begin{lstlisting}[style=python, caption=Python多变量可视化]
# 散点图矩阵
sns.pairplot(data.select_dtypes(include=[np.number]))
plt.show()

# 相关性热力图
plt.figure(figsize=(10, 8))
correlation_matrix = data.select_dtypes(include=[np.number]).corr()
sns.heatmap(correlation_matrix, annot=True, cmap='coolwarm', center=0)
plt.title('变量相关性热力图')
plt.show()
\end{lstlisting}

\subsubsection{R可视化代码}
\begin{lstlisting}[style=R, caption=R可视化代码]
# 直方图
par(mfrow=c(2,2))
for(col in names(data)[sapply(data, is.numeric)]){
    hist(data[[col]], main=paste(col, "分布"), xlab=col)
}

# 箱线图
boxplot(data[sapply(data, is.numeric)], main="数值型变量箱线图")

# 散点图矩阵
pairs(data[sapply(data, is.numeric)])

# 相关性热力图
library(corrplot)
cor_matrix <- cor(data[sapply(data, is.numeric)])
corrplot(cor_matrix, method = "color")
\end{lstlisting}

\subsection{假设检验代码}
\label{app:hypothesis_code}

\subsubsection{Python Mann-Whitney U 检验代码}
\begin{lstlisting}[style=python, caption=Python Mann-Whitney U 检验]
from ucimirepo import fetch_ucirepo
import pandas as pd
from scipy.stats import mannwhitneyu

cdc = fetch_ucirepo(id=891)
df = cdc.data.features.copy()
df['Diabetes_binary'] = cdc.data.targets

group0 = df[df['Diabetes_binary'] == 0]
group1 = df[df['Diabetes_binary'] == 1]

for col in ['BMI', 'MentHlth', 'PhysHlth']:
    u, p = mannwhitneyu(group0[col].dropna(),
                        group1[col].dropna(),
                        alternative='two-sided')
    print(f"{col:<10} | U = {u:>12.2f} | p = {p:>12.6f}")
\end{lstlisting}

\subsection{数据建模代码}
\label{app:logistic_code}

\subsubsection{Python 逻辑回归建模代码}
\begin{lstlisting}[style=python, caption=Python 逻辑回归建模]
from ucimirepo import fetch_ucirepo
import pandas as pd
import numpy as np
import seaborn as sns
import matplotlib.pyplot as plt
from sklearn.linear_model import LogisticRegression
from sklearn.metrics import (classification_report, roc_auc_score,
                             roc_curve, confusion_matrix, accuracy_score)
from sklearn.model_selection import train_test_split

cdc = fetch_ucirepo(id=891)
X = cdc.data.features
y = cdc.data.targets

X_train, X_test, y_train, y_test = train_test_split(
    X, y, test_size=0.25, random_state=42, stratify=y)

logit = LogisticRegression(max_iter=1000)
logit.fit(X_train, y_train)

y_pred = logit.predict(X_test)
y_pred_proba = logit.predict_proba(X_test)[:, 1]

print(classification_report(y_test, y_pred))
print(f"Accuracy: {accuracy_score(y_test, y_pred):.3f}")
print(f"AUC: {roc_auc_score(y_test, y_pred_proba):.3f}")

# ROC 曲线
fpr, tpr, _ = roc_curve(y_test, y_pred_proba)
plt.figure(figsize=(5, 4))
sns.lineplot(x=fpr, y=tpr, label=f"ROC (AUC = {roc_auc_score(y_test, y_pred_proba):.3f})")
plt.plot([0, 1], [0, 1], 'k--')
plt.xlabel('FPR')
plt.ylabel('TPR')
plt.title('ROC Curve')
plt.tight_layout()
plt.show()

# 混淆矩阵
plt.figure(figsize=(4, 3))
sns.heatmap(confusion_matrix(y_test, y_pred),
            annot=True, fmt='d', cmap='Blues',
            xticklabels=['No', 'Diabetes'],
            yticklabels=['No', 'Diabetes'])
plt.xlabel('Predicted')
plt.ylabel('Actual')
plt.title('Confusion Matrix')
plt.tight_layout()
plt.show()

# 提取关键影响因素
coef_df = (pd.DataFrame({'Feature': X.columns,
                         'Coefficient': logit.coef_[0],
                         'OR': np.exp(logit.coef_[0])})
           .assign(Abs_OR=lambda d: d['OR'].abs())
           .sort_values('Abs_OR', ascending=False))
print('\nTop 15 most influential factors (OR)')
print(coef_df.head(15).round(3))
\end{lstlisting}

\section{数据字典}
\label{app:data_dict}

\begin{longtable}{p{4cm}p{2cm}p{8cm}}
\caption{数据字段说明表}\label{tab:data_dictionary} \\
\toprule
\textbf{变量名称} & \textbf{类型} & \textbf{描述与取值范围} \\
\midrule
\endfirsthead

\multicolumn{3}{c}%
{{\bfseries 续表 \thetable\ 数据字段说明表}} \\
\toprule
\textbf{变量名称} & \textbf{类型} & \textbf{描述与取值范围} \\
\midrule
\endhead

\midrule
\multicolumn{3}{r}{{续下页}} \\
\endfoot

\midrule
\endlastfoot

Diabetes\_binary & 二元分类 & 糖尿病诊断结果:0 = 无糖尿病,1 = 有糖尿病 \\
\midrule
HighBP & 二元分类 & 高血压诊断:0 = 无,1 = 有 \\
\midrule
HighChol & 二元分类 & 高胆固醇诊断:0 = 无,1 = 有 \\
\midrule
CholCheck & 二元分类 & 过去5年内是否检查过胆固醇:0 = 否,1 = 是 \\
\midrule
BMI & 连续 & 身体质量指数(Body Mass Index),取值范围:12-98 \\
\midrule
Smoker & 二元分类 & 是否吸烟(至少100支香烟):0 = 否,1 = 是 \\
\midrule
Stroke & 二元分类 & 是否曾患中风:0 = 否,1 = 是 \\
\midrule
HeartDiseaseorAttack & 二元分类 & 是否患有冠心病或心肌梗死:0 = 否,1 = 是 \\
\midrule
PhysActivity & 二元分类 & 过去30天内是否有体育锻炼:0 = 否,1 = 是 \\
\midrule
Fruits & 二元分类 & 每日水果摄入是否≥1次:0 = 否,1 = 是 \\
\midrule
Veggies & 二元分类 & 每日蔬菜摄入是否≥1次:0 = 否,1 = 是 \\
\midrule
HvyAlcoholConsump & 二元分类 & 重度饮酒(男性:每周≥14杯;女性:每周≥7杯):0 = 否,1 = 是 \\
\midrule
AnyHealthcare & 二元分类 & 是否有医疗保险:0 = 否,1 = 是 \\
\midrule
NoDocbcCost & 二元分类 & 是否因费用问题未就医:0 = 否,1 = 是 \\
\midrule
GenHlth & 序数分类 & 总体健康状况:1 = 优秀,2 = 非常好,3 = 好,4 = 一般,5 = 差 \\
\midrule
MentHlth & 连续 & 过去30天心理健康不佳天数,取值范围:0-30 \\
\midrule
PhysHlth & 连续 & 过去30天身体健康不佳天数,取值范围:0-30 \\
\midrule
DiffWalk & 二元分类 & 是否因健康问题行走困难:0 = 否,1 = 是 \\
\midrule
Sex & 二元分类 & 性别:0 = 女性,1 = 男性 \\
\midrule
Age & 次序分类 & 年龄分组:\\
& & 1=18-24, 2=25-29, 3=30-34, 4=35-39, \\
& & 5=40-44, 6=45-49, 7=50-54, 8=55-59,\\
& & 9=60-64, 10=65-69, 11=70-74,\\ 
& & 12=75-79, 13:$\geq $80 \\
\midrule
Education & 次序分类 & 教育程度:\\
& & 1=未上过学,2=小学,3=初中,4=高中,5=大专,6=本科及以上 \\
\midrule
Income & 次序分类 & 家庭年收入(美元):\\
& & 1: $\leq 10k$, 2: $10k-15k$, 3: $15k-20k$,\\
& & 4: $20k-25k$, 5: $25k-35k$,\\
& & 6: $35k-50k$, 7: $50k-75k$, 8: $\geq 75k$ \\
\bottomrule
\end{longtable}

\textbf{数据说明:}
\begin{itemize}
    \item 数据来源于 CDC BRFSS 2021 年度调查,共包含 253,680 条有效记录。
    \item 所有分类变量(二元/序数)均已进行编码处理,便于统计分析。
    \item 连续变量 BMI、MentHlth、PhysHlth 已进行异常值检测与处理。
    \item 数据已清洗,无缺失值,可直接用于建模与分析。
\end{itemize}


\end{document}