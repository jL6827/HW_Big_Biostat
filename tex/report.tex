\documentclass[12pt]{ctexart}
\usepackage[utf8]{inputenc}
\usepackage{graphicx}
\usepackage{float}
\usepackage{booktabs}
\usepackage{amsmath}
\usepackage{hyperref}
\usepackage{listings}
\usepackage{xcolor}
\usepackage{caption}

% 代码样式设置
\definecolor{codegreen}{rgb}{0,0.6,0}
\definecolor{codegray}{rgb}{0.5,0.5,0.5}
\definecolor{codepurple}{rgb}{0.58,0,0.82}
\definecolor{backcolour}{rgb}{0.95,0.95,0.92}

\lstdefinestyle{python}{
    backgroundcolor=\color{backcolour},   
    commentstyle=\color{codegreen},
    keywordstyle=\color{magenta},
    numberstyle=\tiny\color{codegray},
    stringstyle=\color{codepurple},
    basicstyle=\ttfamily\footnotesize,
    breakatwhitespace=false,         
    breaklines=true,                 
    captionpos=b,                    
    keepspaces=true,                 
    numbers=left,                    
    numbersep=5pt,                  
    showspaces=false,                
    showstringspaces=false,
    showtabs=false,                  
    tabsize=2,
    language=Python
}

\lstdefinestyle{R}{
    backgroundcolor=\color{backcolour},   
    commentstyle=\color{codegreen},
    keywordstyle=\color{magenta},
    numberstyle=\tiny\color{codegray},
    stringstyle=\color{codepurple},
    basicstyle=\ttfamily\footnotesize,
    breakatwhitespace=false,         
    breaklines=true,                 
    captionpos=b,                    
    keepspaces=true,                 
    numbers=left,                    
    numbersep=5pt,                  
    showspaces=false,                
    showstringspaces=false,
    showtabs=false,                  
    tabsize=2,
    language=R
}

\title{数据分析报告}
\author{你的名字}
\date{\today}

\begin{document}

\maketitle

\begin{abstract}
本报告对数据集进行了全面的分析,包括描述性统计、可视化分析、假设检验和数据建模。通过Python和R语言相结合的方式,深入挖掘数据特征和规律。
\end{abstract}

\tableofcontents
\newpage

\section{引言}
\subsection{研究背景}
简要介绍研究背景和目的。

\subsection{数据来源}
描述数据来源和数据集的基本信息。

\section{数据预处理}
\subsection{数据加载与清洗}
描述数据预处理的过程和方法,包括缺失值处理、异常值检测等。对数据分析后发现没有缺失值。

\textbf{关键步骤:}
\begin{itemize}
    \item 数据加载和初步探索
    \item 缺失值处理
    \item 异常值检测和处理
    \item 数据格式转换
\end{itemize}

\subsection{数据探索}
展示数据的基本信息和结构。

\section{描述性统计}
\subsection{数值型变量统计}
展示数值型变量的集中趋势、离散程度和分布形态。

\textbf{主要统计量:}
\begin{itemize}
    \item 均值、中位数、众数
    \item 标准差、方差、极差
    \item 偏度、峰度
\end{itemize}

\subsection{分类变量统计}
展示分类变量的频数分布和比例。

\section{可视化分析}
\subsection{单变量分析}
通过直方图、箱线图等展示单个变量的分布特征。

\subsection{多变量分析}
通过散点图矩阵、相关性热力图等展示变量间的关系。

\section{假设检验}
\subsection{正态性检验}
检验数据是否符合正态分布。

\subsection{t检验}
比较两组数据的均值差异。

\subsection{方差分析}
比较多组数据间的均值差异。

\section{数据建模}
\subsection{线性回归}
建立线性回归模型,分析变量间的线性关系。

\subsection{分类模型}
使用逻辑回归等分类算法。

\subsection{聚类分析}
对数据进行聚类分析。

\section{结论与建议}
\subsection{主要发现}
总结分析过程中的主要发现。

\subsection{局限性}
讨论分析的局限性。

\subsection{建议}
基于分析结果提出建议。

\section*{参考文献}
\begin{itemize}
    \item 参考文献1
    \item 参考文献2
\end{itemize}

\appendix
\section{代码附录}

\subsection{数据预处理代码}

\subsubsection{Python数据加载与清洗}
\begin{lstlisting}[style=python, caption=Python数据加载与清洗]
import pandas as pd
import numpy as np
import matplotlib.pyplot as plt
import seaborn as sns

# 加载数据
data = pd.read_csv('your_dataset.csv')

# 数据清洗
# 处理缺失值
data = data.dropna()
# 处理异常值
# ... 其他预处理步骤

# 查看数据基本信息
print(data.info())
print(data.describe())
print(data.head())
\end{lstlisting}

\subsubsection{R数据加载与清洗}
\begin{lstlisting}[style=R, caption=R数据加载与清洗]
# 加载数据
data <- read.csv('your_dataset.csv')

# 数据清洗
# 处理缺失值
data <- na.omit(data)
# 处理异常值
# ... 其他预处理步骤

# 查看数据基本信息
summary(data)
head(data)
str(data)
\end{lstlisting}

\subsection{描述性统计代码}

\subsubsection{Python描述性统计}
\begin{lstlisting}[style=python, caption=Python描述性统计]
# 数值型变量的描述性统计
numeric_stats = data.describe()
print(numeric_stats)

# 计算偏度和峰度
from scipy.stats import skew, kurtosis
for column in data.select_dtypes(include=[np.number]).columns:
    print(f"{column}: 偏度={skew(data[column])}, 峰度={kurtosis(data[column])}")

# 分类变量的频数统计
categorical_stats = data.describe(include=['object'])
print(categorical_stats)

# 各分类变量的频数分布
for column in data.select_dtypes(include=['object']).columns:
    print(f"\n{column}的频数分布:")
    print(data[column].value_counts())
\end{lstlisting}

\subsubsection{R描述性统计}
\begin{lstlisting}[style=R, caption=R描述性统计]
# 数值型变量的描述性统计
summary(data)

# 计算偏度和峰度
library(moments)
for(col in names(data)[sapply(data, is.numeric)]){
    cat(col, ": 偏度=", skewness(data[[col]]), 
        ", 峰度=", kurtosis(data[[col]]), "\n")
}

# 分类变量统计
table(data$categorical_variable)
prop.table(table(data$categorical_variable))
\end{lstlisting}

\subsection{可视化分析代码}

\subsubsection{Python单变量可视化}
\begin{lstlisting}[style=python, caption=Python单变量可视化]
# 数值型变量的直方图
plt.figure(figsize=(15, 10))
for i, column in enumerate(data.select_dtypes(include=[np.number]).columns):
    plt.subplot(3, 3, i+1)
    data[column].hist(bins=30)
    plt.title(f'{column}分布')
plt.tight_layout()
plt.show()

# 箱线图
plt.figure(figsize=(15, 10))
data.select_dtypes(include=[np.number]).boxplot()
plt.title('数值型变量箱线图')
plt.xticks(rotation=45)
plt.show()
\end{lstlisting}

\subsubsection{Python多变量可视化}
\begin{lstlisting}[style=python, caption=Python多变量可视化]
# 散点图矩阵
sns.pairplot(data.select_dtypes(include=[np.number]))
plt.show()

# 相关性热力图
plt.figure(figsize=(10, 8))
correlation_matrix = data.select_dtypes(include=[np.number]).corr()
sns.heatmap(correlation_matrix, annot=True, cmap='coolwarm', center=0)
plt.title('变量相关性热力图')
plt.show()
\end{lstlisting}

\subsubsection{R可视化代码}
\begin{lstlisting}[style=R, caption=R可视化代码]
# 直方图
par(mfrow=c(2,2))
for(col in names(data)[sapply(data, is.numeric)]){
    hist(data[[col]], main=paste(col, "分布"), xlab=col)
}

# 箱线图
boxplot(data[sapply(data, is.numeric)], main="数值型变量箱线图")

# 散点图矩阵
pairs(data[sapply(data, is.numeric)])

# 相关性热力图
library(corrplot)
cor_matrix <- cor(data[sapply(data, is.numeric)])
corrplot(cor_matrix, method = "color")
\end{lstlisting}

\subsection{假设检验代码}

\subsubsection{Python假设检验}
\begin{lstlisting}[style=python, caption=Python假设检验]
from scipy.stats import shapiro, normaltest, ttest_ind, ttest_rel

# 正态性检验
for column in data.select_dtypes(include=[np.number]).columns:
    stat, p = shapiro(data[column])
    print(f'{column}: Shapiro-Wilk检验 p值 = {p:.4f}')

# D'Agostino检验
for column in data.select_dtypes(include=[np.number]).columns:
    stat, p = normaltest(data[column])
    print(f'{column}: D\'Agostino检验 p值 = {p:.4f}')

# 独立样本t检验示例
group1 = data[data['group'] == 'A']['value']
group2 = data[data['group'] == 'B']['value']
t_stat, p_value = ttest_ind(group1, group2)
print(f"独立样本t检验: t统计量={t_stat:.4f}, p值={p_value:.4f}")

# 配对样本t检验示例
t_stat, p_value = ttest_rel(data['before'], data['after'])
print(f"配对样本t检验: t统计量={t_stat:.4f}, p值={p_value:.4f}")
\end{lstlisting}

\subsubsection{R假设检验}
\begin{lstlisting}[style=R, caption=R假设检验]
# 正态性检验
for(col in names(data)[sapply(data, is.numeric)]){
    result <- shapiro.test(data[[col]])
    cat(col, ": Shapiro-Wilk检验 p值 =", result$p.value, "\n")
}

# t检验
t_test_result <- t.test(value ~ group, data=data)
print(t_test_result)

# 方差分析
anova_result <- aov(value ~ group, data=data)
summary(anova_result)

# 多因素方差分析
anova_result2 <- aov(value ~ group1 * group2, data=data)
summary(anova_result2)
\end{lstlisting}

\subsection{数据建模代码}

\subsubsection{Python建模代码}
\begin{lstlisting}[style=python, caption=Python线性回归]
from sklearn.linear_model import LinearRegression, LogisticRegression
from sklearn.model_selection import train_test_split
from sklearn.metrics import mean_squared_error, r2_score, classification_report, confusion_matrix

# 线性回归
X = data[['feature1', 'feature2', 'feature3']]
y = data['target']

X_train, X_test, y_train, y_test = train_test_split(X, y, test_size=0.2, random_state=42)

model = LinearRegression()
model.fit(X_train, y_train)

y_pred = model.predict(X_test)

mse = mean_squared_error(y_test, y_pred)
r2 = r2_score(y_test, y_pred)
print(f"均方误差: {mse:.4f}")
print(f"R²分数: {r2:.4f}")

# 逻辑回归
X_class = data[['feature1', 'feature2']]
y_class = data['class']

X_train_c, X_test_c, y_train_c, y_test_c = train_test_split(X_class, y_class, test_size=0.3, random_state=42)

log_model = LogisticRegression()
log_model.fit(X_train_c, y_train_c)

y_pred_c = log_model.predict(X_test_c)

print("分类报告:")
print(classification_report(y_test_c, y_pred_c))
print("混淆矩阵:")
print(confusion_matrix(y_test_c, y_pred_c))
\end{lstlisting}

\subsubsection{R建模代码}
\begin{lstlisting}[style=R, caption=R建模代码]
# 线性回归
model <- lm(target ~ feature1 + feature2 + feature3, data=data)
summary(model)

# 模型诊断
par(mfrow=c(2,2))
plot(model)

# 逻辑回归
log_model <- glm(class ~ feature1 + feature2, data=data, family=binomial)
summary(log_model)

# 聚类分析
set.seed(123)
kmeans_result <- kmeans(scale(data[sapply(data, is.numeric)]), centers=3)
table(kmeans_result$cluster)

# 可视化聚类结果
library(factoextra)
fviz_cluster(kmeans_result, data = data[sapply(data, is.numeric)])
\end{lstlisting}

\section{数据字典}
提供数据字段的详细说明。

\section{附加图表}
额外的分析图表。

\end{document}